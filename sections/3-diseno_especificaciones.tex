\section{Diseño y Especificaciones} \label{sec:definicion-especificaciones}

\subsection{Definición del problema}\label{subsec:definicion-problema}
La falta de información, junto con la carencia de apropiación y de herramientas y recursos disponibles, impide avanzar de manera efectiva hacia las metas y objetivos necesarios para reparar, mejorar y abordar los problemas relacionados con los ecosistemas y sus componentes. Es por lo anterior que se puede pensar el problema, al cual se enfrentan las empresas actualmente, tres líneas. La primera está relacionada a la misma empresa y su información disponible. En esta línea existen dos posibilidades, o la empresa no desea recolectar información relacionada a las dependencias e impactos sobre el agua o la empresa no tiene los recursos para recolectar esa información. En el primer caso, cuando la empresa no desea recolectar la información a pesar de tener los recursos, puede ser un indicativo que las metas, objetivos, visión o misión de la empresa no están relacionados con el medio ambiente y los ecosistemas. En este caso, es evidente que la empresa tendrá el peor nivel posible como su estado actual. Sin embargo, este no es el verdadero problema ya que, al no tener temas relacionados al medio ambiente o ecosistemas en los componentes de su negocio, hacer una transformación es casi que imposible. Si la empresa no tiene la intención, interés o incentivos para hacer un cambio, la estrategia que se quiere plantear en este documento no será de verdadera ayuda. Solamente cuando una empresa tiene incentivos para hacer cambios los hará. Es por esto por lo que metas como la Meta 15 o la definición de Taxonomía Verde creada por el Ministerio de Hacienda son fundamentales para incentivar cambios en las empresas. El segundo caso, en cuanto a la información disponible, es cuando la empresa no tiene los recursos, ya sean monetarios o de capital, para hacer seguimiento a las dependencias e impactos que tiene sobre un capital natural. En este caso, la solución más realista sería la utilización de estrategias o herramientas como las que se buscan plantear en este documento. Sin embargo, dado que actualmente estas no existen o tiene contextos diferentes a los colombianos, las empresas realmente no pueden utilizarlas. La segunda línea de problema que enfrentan las empresas es que aquellas empresas que si tienen información no la relacionan con los ecosistemas, o en este caso con el agua. Cuando una empresa no es consiente que existe una relación entre su negocio y el agua (o cualquier otro capital natural), la posibilidad de cambio en el corto o largo plazo es casi que inexistente. Así mismo, si las empresas no son conscientes de las numerosas interrelaciones entre sus negocios y los ecosistemas, comienzan a generar impactos y afectaciones cada vez más graves. Esta línea es un problema que demuestra la verdadera cara de una empresa con respecto al problema general. Finalmente, la tercera línea está relacionada a la falta de herramientas que se puedan utilizar para determinar el nivel actual de sus dependencias e impactos sobre el agua (o cualquier otro capital natural). A pesar de que en la sección \nameref{subsec:antecedentes}, se mencionaron herramientas disponibles para que una empresa puede usar, también se resaltaron los problemas de estas. El problema principal que reúne a todas las herramientas disponibles ya mencionadas es que estas consideran pequeñas partes de un rompecabezas compuesto por miles de piezas. Algunas herramientas solo consideran el contexto de la empresa, otras solo consideran la ubicación de esta, otras hacen grandes supuestos, etc. Entonces, las empresas no pueden tener una visión completa y real de cuál es su situación actual. Siempre existirá algo que haga falta tener en cuenta, o algún pedazo de información que no se analizó. Las diferencias y necesidades entre cada empresa de un mismo país, de un mismo sector, de una misma industria son significativas. Por ello, desarrollar una estrategia tiene más peso que simplemente crear una herramienta. Una estrategia les permitirá a las empresas evaluar sus necesidades y su información de la manera que ellas más requieran. 

\subsection{Especificaciones} \label{subsec:especificaciones}
Debido a la naturaleza de este documento, las especificaciones se dividirán en los dos objetivos principales. De esta manera, se podrá comprender mejor lo que realmente se desea lograr. Los requerimientos estarán clasificados en requerimientos funcionales y no funcionales.

\subsubsection{Creación de estrategia}
El primer objetivo es crear una estrategia para determinar el nivel de madurez de una empresa en cuanto a sus conocimientos y acciones sobre las dependencias e impactos que tiene sobre el agua como capital natural. En la siguiente tabla se podrá ver los requerimientos funcionales y no funcionales asociados al anterior objetivo.

\begin{table}[H]
    \centering
    \begin{tabular}{p{2cm} | p{2.5cm} | p{10cm}}
        \centering\textbf{Número} & \centering\textbf{Tipo}  & \textbf{Requerimiento} \\
        \hline\hline
        O1-F1 & Funcional & Se selecciona una estrategia que tiene los diferentes niveles de madurez correctamente definidos, especificando el puntaje mínimo y máximo necesario para calificar cada nivel. \\
        \hline
        O1-F2 & Funcional & Las dimensiones que definen el nivel de madurez de la empresa cubren todos los aspectos relacionados con la evaluación de una empresa y sus relaciones con el agua. \\
        \hline
        O1-F3 & Funcional & Se define el peso que cada dimensión debe tener para la determinación del nivel de madurez. \\
        \hline
        O1-NF1 & No Funcional & La estrategia es fácil de entender para cualquier empresa que desee utilizarla. \\
        \hline
        O1-NF2 & No Funcional & La estrategia debe ser flexible y adaptable a diferentes contextos empresariales y condiciones ambientales, permitiendo su aplicación en una amplia variedad de situaciones. \\
        \noalign{\global\arrayrulewidth=1pt} 
        \hline
    \end{tabular}
    \caption{Requerimientos asociados al primer objetivo}
    \label{tab:rqm-obj1}
\end{table}

\subsubsection{Desarrollo del modelo}
El segundo objetivo es desarrollar e implementar el modelo generado por la estrategia en una herramienta que permita determinar el nivel de madurez de la empresa, al mismo tiempo que se visualiza su relación con el agua a partir de la información proporcionada por la misma empresa. En la siguiente tabla se podrá ver los requerimientos funcionales y no funcionales asociados al anterior objetivo.

\begin{table}[H]
    \centering
    \begin{tabular}{p{2cm} | p{2.5cm} | p{10cm}}
        \centering\textbf{Número} & \centering\textbf{Tipo}  & \textbf{Requerimiento} \\
        \hline\hline
        O2-F1 & Funcional & La herramienta proporciona suficiente información y claridad para que la empresa sepa y entienda su estado actual en términos de su nivel de madurez. \\
        \hline
        O2-F2 & Funcional & La herramienta presenta el estado actual y la definición de cada dimensión, permitiendo que la empresa identifique el orden de las dimensiones desde la mejor hasta la peor.\\
        \hline
        O2-F3 & Funcional & La herramienta muestra características descriptivas de los indicadores relacionados con el agua seleccionados por la empresa. \\
        \hline
        O2-F4 & Funcional & La herramienta expone a la empresa cuáles son sus relaciones con el agua como capital natural, basadas en la información que ella misma proporciona. \\
        \hline
        O2-NF1 & No Funcional & La herramienta debe contar con una interfaz de usuario intuitiva y amigable que facilite la interacción de los usuarios, permitiendo una fácil comprensión y navegación, independientemente del nivel de experiencia técnica. \\
        \hline
        O2-NF2 & No Funcional & La herramienta debe ser diseñada de manera que no se requiera la contratación de un experto en temas ecosistémicos para su utilización.\\
        \hline
        O2-NF3 & No Funcional & Debe ser accesible para usuarios con diferentes niveles de conocimiento, proporcionando guías claras y recursos de ayuda que permitan a cualquier persona comprender y utilizar eficazmente todas las funcionalidades de la herramienta. \\
        \noalign{\global\arrayrulewidth=1pt} 
        \hline
    \end{tabular}
    \caption{Requerimientos asociados al segundo objetivo}
    \label{tab:rqm-obj2}
\end{table}

\subsubsection{Soluciones inexactas o incompletas}
Cuando se desarrolla una estrategia para evaluar el nivel de madurez de una empresa en relación con su gestión del agua como capital natural, las soluciones inexactas o incompletas pueden surgir de enfoques poco definidos o de la falta de consideración de todas las dimensiones relevantes. Una estrategia mal diseñada podría pasar por alto aspectos críticos o no capturar la complejidad de las interacciones entre la empresa y sus recursos hídricos. Esto podría resultar en una evaluación superficial que no refleje con precisión la situación real de la empresa en términos de sus conocimientos y acciones relacionadas con el agua. Además, la falta de claridad en los criterios de evaluación o la ausencia de métricas específicas podría llevar a interpretaciones erróneas o a decisiones estratégicas poco fundamentadas. Es esencial que la estrategia diseñada sea exhaustiva, precisa y adaptable, incorporando tanto los aspectos técnicos como los contextuales para ofrecer una evaluación completa y fiable del nivel de madurez de la empresa en relación con el agua como capital natural. 

\hfill

Por otro lado, las soluciones inexactas o incompletas en cuanto a la implementación del modelo pueden surgir cuando las herramientas carecen de claridad en sus funciones o no proporcionan suficiente orientación para los usuarios. Esto puede conducir a interpretaciones erróneas de los datos o a decisiones poco informadas. Además, la falta de accesibilidad para usuarios sin experiencia en temas ecosistémicos puede limitar la efectividad de la herramienta y llevar a resultados sesgados o poco precisos. Es crucial que las soluciones implementadas aborden estas preocupaciones, garantizando una presentación clara de la información, una interfaz intuitiva y recursos de apoyo adecuados para que los usuarios puedan aprovechar al máximo la herramienta sin necesidad de contar con la asistencia de expertos externos.


\subsection{Restricciones}
A pesar de que este proyecto busca desarrollar la mejor estrategia, así como la mejor herramienta para ayudar a las empresas. Existen ciertas restricciones que imposibilitan obtener la mejor estrategia y herramienta. La primera restricción a la cual enfrenta la estrategia y, en especial, la herramienta es que la cantidad de personalización perfecta a una empresa es irreal. Dado que cada empresa es tan diferente en casi todos los componentes, tener en cuenta el 100\% de la información de una empresa, así como el 100\% del funcionamiento de una empresa es imposible. Con un ejemplo tan básico como los indicadores que una empresa mide son suficiente para exponer el punto anterior. Los indicadores, a pesar de tener ciertos estándares como valores a medir, rangos y definiciones, no todas las empresas los miden de igual manera. No todas las empresas utilizan las mismas unidades, así como no todas las empresas mide los indicadores en la misma época. Hay empresas que miden los indicadores de manera cualitativa, mientras que otras de manera cuantitativa. Otro ejemplo similar son las iniciativas o programas ecosistémicos que implementa cada empresa. Una empresa pueda realizar n número de iniciativas a favor del agua o de algún servicio ecosistémico asociado al agua. Pero si la estrategia o herramienta no contempla esa iniciativa, se podría malinterpretar las gestiones de la empresa.

\hfill

Otra restricción que se debe tener en cuenta para este proyecto es la falta de estandarización que existen en temas relacionados con capital natural, \acrshort{ssee} y funciones ecosistémicas. A pesar de que estos tres componentes de un ecosistema están relacionados, no hay una guía o marco de referencia que permita, con certeza, relacionarlos. Es decir, no hay una fuente 100\% confiable que relacione todos los capitales naturales con sus respectivo \acrshort{ssee} y funciones que se pueda utilizar de manera estándar. Esta falta de rigurosidad genera que la mayoría de las fuentes de información utilizadas y generadas provengan de opiniones de expertos. Esto puede generar niveles de sesgo e inconsistencia que pueden llegar a generar resultados erróneos o falsos que perjudican la toma de decisión de aquellas empresas que utilicen las herramientas. Para solucionar este problema, se deben utilizar metodologías adicionales cuando se estén haciendo trabajos basados en opiniones de expertos. En el caso de este proyecto, en los momentos que intervinieron expertos se utilizaron guías mundialmente referenciadas o metodologías externas como la metodología \textit{Analytic Hierarchy Process} para determinar importancias relativas. 

\hfill

Finalmente, la última restricción que tiene este proyecto está relacionada con la herramienta desarrollada del modelo generado por la estrategia. Cuando una empresa utilicé la herramienta la necesidad de un pedir asistencia a un experto, por fuera de la empresa, se debe buscar minimizar. Si la herramienta necesitase que la empresa contrate a un experto para que la utilice, la empresa debería mejor contratar un servicio de asesoría directamente. Adicionalmente, si la herramienta necesitara de un experto presente, la herramienta no estaría dirigida a las empresas sino estaría dirigida como un servicio que podría ofrecer las compañías de asesoramiento ecológico. Finalmente, la importancia y las implicaciones positivas que podría traer la herramienta dejarían de existir ya que la empresa no estaría involucrada en su propio cambio y no generaría esa concientización deseada.