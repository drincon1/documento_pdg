\section{Descripción General} \label{sec:descripcion-general}

\subsection{Objetivos} \label{subsec:objetivos}
La Conferencia de las Naciones Unidas sobre la Diversidad Biológica (COP 15) terminó el 19 de diciembre de 2022 con un acuerdo cuyo propósito era orientar las acciones de cada país a favor de la naturaleza hasta el 2030. En dicha conferencia se establecieron diferentes metas para lograr los objetivos establecidos. Entre estas metas se destaca la llamada Meta 15, cuyo objetivo era que las empresas deberán supervisar, evaluar y divulgar periódicamente sus riesgos, dependencias e impactos sobre la biodiversidad para el 2030. Se espera que las empresas generen reportes periódicos que estén relacionados con los servicios ecosistémicos de los cuales se benefician. Con estos reportes la COP 15 busca que las empresas a nivel mundial no solo se concienticen de sus impactos a los ecosistemas, sino que también se generen reparaciones e inversiones a favor de los ecosistemas.

\hfill

Ceres, por otro lado, es una organización sin ánimo de lucro cuya misión es “ayudar a los inversores y las empresas a acelerar la transición hacia una economía más limpia, justa y sostenible” \parencite{ceres-no-date}. \textit{Valuing Water Finance Initiative Benchmark}, es uno de los proyectos que realiza esta organización cuyo propósito es que “[…] las empresas reconocerán el agua dulce como el agua natural más preciada del mundo esencial para industrias enteras y para todas las comunidades y ecosistemas” \parencite{ceres-2023B}.

\hfill

La Meta 15 y el proyecto \textit{Valuing Water Finance Initiative Benchmark} demuestran el contexto en el cual se trabaja el proyecto que se presenta en este documento. Las empresas necesitan saber su estado actual en diversos temas alrededor de los ecosistemas y el medio ambiente. Ellas requieren de eso ya que necesitan empezar a cumplir con ciertos requerimientos o metas para no infligir ninguna ley y tener costos inesperados. Esto solo lo pueden evitar sabiendo su estado actual y para esto requieren de herramientas y estrategias. El proyecto de Ceres previamente mencionado es uno de los pocos proyectos que actualmente ayuda a las empresas a determinar su estado. Sin embargo, es un proyecto que fue hecho a “puertas cerradas”, es decir, fue Ceres quien busco las empresas y su información. Por lo tanto, una empresa no podía ser evaluada si quería ya que Ceres determinaba las empresas a evaluar. Adicionalmente, dado que Ceres se basaba en la información pública, no toda la información de una empresa fue tomada en consideración. En especial, las necesidades precisas que una empresa tiene bajo su propio contexto. Aunque el proyecto de Ceres es excelente y sus resultados son bastante enriquecedores, el proyecto se limita a un número de empresas donde no toda la información es contemplada ni las necesidades específicas de cada empresa son tomadas en cuenta. 

\hfill

El contexto anterior nos permite definir el propósito de este proyecto. Este proyecto tiene dos objetivos principales los cuales se dividen en la creación de la estrategia y en la implementación del modelo generado por la estrategia. Esta distinción es necesaria porque los parámetros evaluados en cada diferente estrategia generaban diferentes modelos. El primer objetivo es crear una estrategia para determinar el nivel de madurez de una empresa en cuanto a sus conocimientos y acciones sobre las dependencias e impactos que tiene sobre el agua como capital natural. El segundo objetivo es desarrollar e implementar el modelo generado por la estrategia en una herramienta que permita determinar el nivel de madurez de la empresa, al mismo tiempo que se visualiza su relación con el agua a partir de la información proporcionada por la misma empresa. 

\subsection{Antecedentes} \label{subsec:antecedentes}
El concepto de crear una estrategia para que las empresas puedan determinar su estado no está definido. Las herramientas y literatura que existe en la actualidad sobre construir una estrategia están orientadas a seguir estándares o guías que en muchos casos son muy genéricas o difíciles de entender. Sin embargo, de estos estándares o guías se pueda encontrar mucha información valiosa que permite la construcción de una estrategia bastante robusta y dinámica.  Algunos ejemplos son el \textit{Capital Natural Protocol}, el \textit{CDSB Framework Application guidance for water-related disclosures}, el estándar GRI 303: Agua y efluentes 2018, el \textit{LEAP Approach del TNFD}, entre otros. De todas estas guías, estándares o marcos de referencias se hablará en más detalle en la sección \nameref{subsec:recoleccion-informacion}.

\hfill

En esta sección se hablará de las herramientas actuales que existen y permiten a las empresas conocer su estado actual. Estas herramientas son \textit{\acrlong{encore}} (\acrshort{encore}), \textit{WWF Water Risk Filter} y Ceres \textit{Valuing Water Finance Initiative Benchmark}. Aunque existen más herramientas que buscan exponerle a las empresas el estado actual con respecto a los ecosistemas y los capitales naturales, las herramientas ya mencionadas demuestran los aspectos positivos y los negativos más generales que existen en la actualidad.

\subsubsection{ENCORE} \label{subsubsec:encore}
\textit{\acrlong{encore}} (\acrshort{encore}) es una herramienta que, basada únicamente en la industria y el sector de la empresa, determina los \acrshort{ssee} de los cuales, probablemente, depende más y en los que tiene un mayor impacto la empresa. Aunque esta herramienta expone la relación entre empresa y ecosistema de una manera fácil de entender, no provee mucha información adicional. Es una herramienta que, además, no es personalizable ni configurable al contexto de la empresa. No se contempla las líneas de negocio, la cadena de suministro, la ubicación geográfica, los esfuerzos actuales. Es por eso que la herramienta \acrshort{encore}, a pesar de tener una buena información, es poco útil para una empresa ya que no puede personalizar sus necesidades y tampoco puede obtener información más allá de cuáles son los \acrshort{ssee} que probablemente mayor dependencia tenga y mayor impacto tenga. 


\subsubsection{WWF \textit{Water Risk Filter}} \label{subsubsec:wwf-water-risk-filter}
\acrshort{wwf} \textit{Water Risk Filter} es una herramienta creada por la organización mundial \textit{\acrlong{wwf}} (\acrshort{wwf}) que permite  “[…] a las empresas e inversores evaluar y responder a sus riesgos de agua tanto ahora como en el futuro” \parencite{world-wildlife-fund-2023}. Esta herramienta se basa principalmente en la ubicación geográfica de la empresa para determinar cuáles son los riesgos potenciales de las cuencas y cuales son algunos posibles riesgos operacionales. \acrshort{wwf} \textit{Water Risk Filter} utiliza análisis de escenarios para poder determinar los posibles riesgos que tiene la empresa. Esta herramienta contempla un escenario optimista, un escenario con las tendencias actuales y, por último, un escenario pesimista. En el documento \textit{Water Risk Scenarios} \parencite{world-wildlife-fund-2020} la organización dice lo siguiente,


\hfill
\par
\leftskip=0.35in \rightskip=0.35in
\textit{In a world of uncertainty, scenario analysis is a useful approach for a forward-looking assessment of risks and opportunities, so that businesses can evaluate their resilience under different possible futures.}

\hfill
\par
\leftskip=0in \rightskip=0in

Aunque esto es verdad, el análisis de escenario para determinar riesgos es altamente dependiente de los supuestos hechos por la herramienta y la calidad de la información. Ya que la herramienta solo considera un pedazo de información para hacer el análisis de riesgos, esto podría llenar a conclusiones incorrectas. Ya que la herramienta no contempla diferentes políticas, acciones o procesos a favor del agua que pueda estar haciendo una empresa, se podrían llegar a conclusiones irreales que desmotivaran a la empresa o harían que la empresa incurriera en mayores costos innecesarios. Cuando se quiere hacer una herramienta de riesgos y oportunidades como  \acrshort{wwf} \textit{Water Risk Filter} se debe tener bastante precaución con la cantidad información que se solicita. De igual manera, se debe tener cuidado con el número y el tipo de supuestos que se realizan. No todas las empresas operan de igual manera en mismas ubicaciones geográficas. Esto implica, que se puede correr el riesgo de hacer supuestos muy poco estrictos o con una flexibilidad muy baja.



\subsubsection{Ceres \textit{Valuing Water Finance Initiative Benchmark}} \label{subsubsec:ceres-benchmark}
A pesar de que Ceres \textit{Valuing Water Finance Initiative Benchmark} no es una herramienta abierta al público, si es una herramienta implementada por la organización Ceres. Esta herramienta ya fue discutida en la sección \nameref{subsec:objetivos}, sin embargo se quiere hacer énfasis en su formato de evaluación. La metodología y modelo resultante de este proyecto utilizo varios componentes del proyecto \textit{Valuing Water Finance Initiative Benchmark}, los cuales se explicarán más adelante. El motivo principal es la simpleza pero rigurosidad que utiliza Ceres en el proyecto. En el documento \textit{Valuing Water Finance Initiative Benchmark} \parencite{ceres-2023B} se expone la forma que se utilizó para puntuar a cada empresa según, lo que la organización denomina, expectativas. A través de seis (6) expectativas se calificaba a cada empresa con un puntaje máximo de 15 puntos por expectativa. Estos 15 puntos estaban repartidos en dos indicadores principales cada uno con 5 puntos y los restante cinco puntos con métricas adicionales. Una empresa, por lo tanto, podría llegar a tener un máximo de 90 puntos. Una empresa se calificaba en un nivel según el porcentaje de puntos obtenidos. En el marco de referencia hecho por Ceres existían solo cuatro niveles disponibles. Como dicho previamente, el principal problema con este proyecto es que una empresa no podía hacer el ejercicio sino que la organización escogía a las empresas. Adicionalmente, no se utilizaba toda la información que una empresa tenía sino solamente la información pública que encontraba la organización. Otro problema adicional, es que la empresa considera que las seis expectativas tiene el mismo peso. A pesar de que cada expectativa es importante, claramente existen expectativas más importantes que otras en el contexto del agua, en especial, si se quiere concientizar a las empresas sobre la importancia del agua como capital natural. Al considerar que todas las expectativas tienen el mismo peso, una empresa que pueda estar haciendo todo su esfuerzo por mejorar la calidad del agua (Water Quality, primera expectativa) y disminuir la cantidad de agua que utiliza (Water Quantity, segunda expectativa), pero no está teniendo un gran desempeño en políticas públicas (Public Policy Engagemente, sexta expectativa) se verían perjudica en el puntaje total. Bajo la intención de darle al agua una mayor importancia, considerar las tres expectativas como iguales es un posible error.

\subsection{Identificación del problema y de su importancia} \label{subsec:identificacion-problema-importancia}
La importancia de este proyecto va más allá de encontrar una estrategia y un modelo para determinar el estado actual de una empresa con respecto al agua. La importancia real de este proyecto es encontrar una manera fácil, económica, exequible y flexible de encontrar la relación entre un negocio y un capital natural cualquiera. Concientizar a una empresa de que las acciones y decisiones que toma en su día a día tienen efectos y vínculos con casi todos los capitales naturales, \acrshort{ssee} y funciones ecosistémicas es la verdadera importancia de este proyecto. Cuando una empresa empieza a ser consciente de que sus operaciones van más allá de lo operacional y monetario, el propósito real de este proyecto se empieza a cumplir. 

\hfill

Es ideal que a través de este proyecto cualquier empresa pudiera armar su propia estrategia y modelo con las dimensiones más relevantes que esta considere para determinar su nivel actual con respecto a uso de un capital natural. Por eso, este proyecto tiene la intención de no solo mostrar cómo se llegó a un estrategia y modelo, sino también mostrar todos los recursos que existen actualmente. En especial en el caso colombiano donde las empresas realmente carecen de herramientas, guías o estándares para poder realizar este ejercicio. Las empresas de este país no pueden realizar este ejercicio con tanta facilidad ya sea porque los recursos disponibles de entidades gubernamentales tienen varias carencias o porque las guías o marcos de referencias están más asociados a contextos americanos o europeos. Por ejemplo, el \acrfull{siac} constantemente tiene problemas de funcionamiento. Las páginas se demoran mucho tiempo en cargar, la información aparece incompleta y es difícil de leer la información en algunos casos ya que no carga por completo.  Otro ejemplo, es la página de BioModelos realizada por el Instituto Humbolt, en la cual se puede observar que en el tema de especies la falta de información es bastante alta y la cantidad de modelos válidos es mínima. Por otro lado, existen protocolos o guías como la \textit{Capital Natural Protocol}, que a pesar de ser muy buena y tener bastante información, no es fácil de aplicar en un escenario colombiano. Un ejemplo más claro se podría ver en el número de indicadores disponibles registrados de manera oficial en Europa versus en Colombia. En el \acrfull{ideam} para mayo del 2024 existen once indicadores relacionados con el agua. En cambio, en el reporte del año 2003 de la \textit{European Environment Agency “An inventory of biodiversity indicators in Europe”} tiene cuarenta y tres indicadores relacionados solo al agua. Ejemplos como el anterior, relevan las tendencias actuales a las que se tuvo que enfrentar este proyecto en su desarrollo. Es por esto que, a través de este proyecto, se busca aumentar la cantidad de información que una empresa debería tener en cuenta y dónde podría encontrar esta información.

\hfill

Este proyecto está relacionado temáticas globales como, por ejemplo, la ya mencionada COP 15. Adicionalmente, de manera indirecta, este proyecto está también relacionado con el \acrshort{sdg6} creada por las Naciones Unidas en el año 2015.  El \textit{\acrfull{sdg6}} tiene como objetivo “garantizar la disponibilidad y la gestión sostenible del agua y el saneamiento para todos” \parencite{world-health-organization-2017}. A través del proyecto de este documento, se busca que las empresas puedan hacer cambios para cumplir con el \acrshort{sdg6}. Cabe a aclar que el objetivo final de este proyecto, en ningún momento es que la estrategia planteada y su herramienta hagan que las empresas cumplan con el \acrshort{sdg6}. Simplemente, la implementación de una estrategia como la planteada en este proyecto y su modelo pueden ser una externalidad positiva que ayude a las empresas a estar orientadas y sincronizadas con objetivos como el del \acrshort{sdg6}. 