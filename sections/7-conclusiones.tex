\section{Conclusiones}
\subsection{Discusión}
En este documento se llegó a una estrategia para determinar el nivel de madurez de una empresa en cuanto a sus conocimientos y acciones sobre las dependencias e impactos que tiene sobre el agua como capital natural. A través del uso de la metodología \textit{benchmark} y siguiendo el Diagrama 4 se logró crear una estrategia que evalúa las dependencias e impactos que tiene a nivel general la empresa sobre el agua como capital natural. Adicionalmente, se obtuvo el modelo de madurez resultado de la estrategia. Modelo de madurez que evalúa siete diferentes dimensiones de la empresa para calificarle en uno de los cuatro niveles disponibles. Además, mediante el trabajo y la opinión de los expertos, se determinaron las relaciones entre los indicadores medidos o considerados por la empresa y los servicios ecosistémicos, funciones y gestiones vinculadas al agua como capital natural. El segundo resultado de este documento está asociado a desarrollar e implementar el modelo generado por la estrategia en una herramienta que permita determinar el nivel de madurez de la empresa, al mismo tiempo que se visualiza su relación con el agua a partir de la información proporcionada por la misma empresa. A través de la implementación de la aplicación web utilizando los servicios descritos en secciones anteriores, se logró mostrar las relaciones entre el negocio y el agua como capital natural, además de mostrar el nivel de madurez y el puntaje obtenido por cada dimensión evaluada. 
En este documento también se presentaron dos estrategias que fueron contempladas, pero debido a sus problemas no fueron seleccionadas. Los problemas principales fueron la falta de información, rigurosidad y exceso de opinión. Estos tres problemas, a pesar de estar minimizados en la estrategia propuesta en este documento, sigue existiendo. Se recomienda a cada empresa seguir investigando y buscando herramientas que les permita ilustrar su situación de manera más específica. Esto con la intención de maximizar la información externa al tiempo que se minimiza la información no utiliza de la empresa. Es fundamental destacar la importancia de mantener la rigurosidad en todo momento, lo cual se logra mediante el uso de guías, estándares o marcos de referencia ampliamente aceptados y utilizados a nivel mundial. Esto garantiza la obtención de la mejor información posible y ayuda a reducir la influencia de opiniones sesgadas o inconsistencias en el proceso de evaluación.

\subsection{Trabajo futuro}
El final de esta estrategia y su herramienta parecen no existir. El desarrollo y profundización de estas dos podría, y debería, seguir cambiando y mejorando año tras año. A continuación, se ilustran tres ejemplos que trabajos futuros que se podrían hacer, ya sea con la estrategia o la herramienta. Explorar la posibilidad de ampliar el enfoque del proyecto para abordar otro capital natural distinto al agua, o incluso trabajar en profundidad un servicio ecosistémico específico. Esto implicaría adaptar la metodología desarrollada para evaluar y comprender la relación de las empresas con este nuevo capital natural o servicio ecosistémico, lo que ampliaría el alcance y la aplicabilidad del proyecto a diferentes contextos y problemáticas ambientales. Por otro lado, también se podría extender el alcance del trabajo para incluir un análisis detallado de los riesgos y oportunidades que enfrentan las empresas en relación con el uso del agua como capital natural. Esto implicaría identificar y evaluar los riesgos financieros, legales, operativos y de reputación asociados con las prácticas de gestión del agua de las empresas, así como las oportunidades de mejora y crecimiento que podrían surgir de una gestión sostenible del recurso hídrico. Finalmente, se podría desarrollar una versión mejorada de la herramienta de evaluación que permita recopilar información de múltiples empresas a nivel nacional y comprender la situación actual del país en términos de dependencia e impacto sobre el agua como capital natural. Esto implicaría la implementación de una plataforma o sistema más robusto y escalable que facilite la recopilación, análisis y visualización de datos a gran escala, lo que proporcionaría una visión más completa y detallada de la gestión del agua en el contexto nacional y permitiría identificar áreas de intervención prioritarias para políticas públicas y acciones empresariales.
