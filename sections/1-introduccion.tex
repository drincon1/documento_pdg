\section{Introducción}
\label{sec:introduccion}
En la actualidad, el mundo se está enfrentando a un reto que el mismo ha creado y que debe solucionar de manera urgente. La utilización del agua como capital natural ha sido un tema que ha cobrado grande importancia en el mundo y a partir de la anterior década se han empezado a fijar objetivos y metas para solucionar el problema. De igual manera, la creación de guías, estándares y marcos de referencias alrededor de capitales naturales como el agua, ha crecido debido a la necesidad de la empresa de entender cuál es su situación actual. En el caso colombiano, las empresas han empezado a desarrollar un mayor interés en estos temas debido a las oportunidades y riesgos financieros que se están creado.  Por ejemplo, a principios del año 2023 ante las nuevas metas y objetivos globales relacionado con los ecosistemas, el Gobierno Nacional empezó a generar sus propias metas y definiciones. El \textcite{ministerio-de-hacienda-2023}, definió el concepto de Taxonomía Verde de la siguiente manera:

\hfill
\par
\leftskip=0.35in \rightskip=0.35in
La Taxonomía Verde de Colombia es un sistema de calificación para actividades y activos económicos con contribuciones sustanciales para el logro de objetivos ambientales, que responden a los compromisos, estrategias y políticas trazados por el Gobierno colombiano en materia ambiental, es decir, define qué es una inversión verde en Colombia.

\hfill
\par
\leftskip=0in \rightskip=0in
Este concepto introduce a las empresas nuevas necesidades y demandas por cumplir con nuevas metas ecológica en busca de calificar sus inversiones y actividades económicas como inversiones verdes. Esto les permitirá el acceso a créditos especiales, inversiones nacionales y extranjeras y, por otro lado, les salvará de pagar multas, penalizaciones o reparaciones que afecten sus estados financieros. Sin embargo, actualmente no existen herramientas que le permitan a las empresas colombianas saber el nivel de dependencia e impacto que tienen sobre el agua como capital natural, dada las acciones y decisiones que toman en sus diferentes procesos y líneas de negocio.

\hfill

El problema al cual se enfrentan las empresas actualmente se puede pensar en tres líneas. La primera está relacionada a la misma empresa y su información disponible. La falta de información ya sea por decisión propia o por falta de recursos, genera incumplimiento de objetivos relacionados al agua y, por lo tanto, aumentan los riesgos de penalizaciones financieras. La segunda línea de problema que enfrentan las empresas es que aquellas empresas que si tienen información no la relacionan con los ecosistemas, o en este caso con el agua. Esto genera que no haya un avance real, dado que demuestra que no hay una concientización o una intención real por mejorar la situación actual. La tercera línea está relacionada a lo mencionado en el párrafo anterior. Actualmente no existe una herramienta que las empresas puedan utilizar para determinar el nivel actual de sus dependencias e impactos sobre el agua. Es por todo lo anterior que la solución deseada debería ser un proceso que busque entender el negocio y sus operaciones, al tiempo que relaciona los impactos y dependencias que tiene la empresa sobre el agua para que así esta pueda saber su estado actual sin la necesidad de un esfuerzo económico significativo ni la contratación externa de expertos.

\hfill

En este documento se presenta un proyecto que busca desarrollar una estrategia para que las empresas puedan determinar su nivel de madurez en cuanto a sus conocimientos y acciones sobre las dependencias e impactos que tiene sobre el agua como capital natural. Para esto, se implementó una metodología similar a la metodología de benchmarking, la cual basándose en los objetivos, se hizo una recolección de diversas fuentes de información para seleccionar y armar el marco de referencia que mejor se adapte a los objetivos. Luego se identificado las mejores dimensiones (KPIs) para evaluar a la empresa y complementar el marco de referencia. Finalmente, después de varias iteraciones para encontrar el mejor modelo, se llegó a la construcción del modelo deseado para así poder obtener el nivel de madurez.  Adicionalmente, a través de la ayuda de expertos en el tema del agua, se desarrolló un ejercicio simple que busca mostrarle a la empresa que relaciones existen entre su negocio y el agua. Especialmente, la relación entre los \acrfull{ssee} relacionados con el agua y los indicadores actualmente considerados por la empresa en relación a este capital natural. 

\hfill

Aplicando la metodología descrita en el párrafo anterior, se llegó a un modelo que se llama Negocio-Agua, el cual tiene como principal objetivo demostrarle a la empresa como se relacionan sus operaciones, características, líneas de negocio, etc. con el agua como capital natural y sus \acrshort{ssee}. Este modelo, al ser un modelo de madurez, contempla siete (7) dimensiones diferentes que ayudan a no solo evaluar todos los aspectos importantes de una empresa en su relación con el agua, sino también a determinar el nivel de madurez actual de la misma. La suma del puntaje obtenido en cada dimensión, luego de una ponderación por importancia relativa hecha por expertos utilizando la metodología \textit{Analytic Hierarchy Process} \parencite{bahurmoz-2006} determina en cuál de los cuatro (4) niveles de madurez posibles está la empresa. Este modelo de madurez se desarrolló en una aplicación web, la cual tiene como eje principal un cuestionario que busca evaluar cada una de las dimensiones. Además, esta herramienta también solicita información a la empresa para poder generar esas relaciones con el agua. Así, luego de completar y llenar toda la información requerida en la herramienta, la empresa podrá saber cuál es su actual nivel de madurez, su mejor dimensión y sus relaciones con el agua basadas en los indicadores que contempla dentro de sus operaciones. 

\hfill

Este documento se divide en seis secciones. En la primera, se realiza una descripción general del proyecto. En esa sección se fijan los objetivos, se nombran los antecedentes y se menciona la importancia del proyecto. En la segunda sección, se define el problema que enfrenta el proyecto, se establecen los requerimientos y las restricciones. La tercera sección menciona el proceso de diseño del proyecto. Se describe la metodología implementada y la estrategia genérica del proyecto. También se describen las fuentes de información para el desarrollo del diseño, así como las alternativas de diseño que se contemplan y las razones por las cuales se rechazan. En la cuarta sección se describe en detalle la solución al problema y se detalla cada etapa del proyecto. En los resultados esperados se menciona la implementación de la herramienta, sus componentes y los resultados de la misma. En la quinta sección se discuten los métodos y resultados de la validación. Se describe cuál es el método de validación y se discuten los resultados obtenidos de la validación. Finalmente, en la última sección se resume todo el trabajo, se nombran los problemas encontrados y cómo se pueden solucionar. Además, se mencionan posibles extensiones y trabajos futuros que podrían derivarse del proyecto actual.