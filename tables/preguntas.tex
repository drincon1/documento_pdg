\begin{longtable}{p{0.75cm}|p{6cm}|p{6cm}|p{2cm}}
\textbf{ID} & \textbf{Preguntas} & \textbf{Opciones} & \textbf{Fuente}  \\
\hline\hline
1 & ¿Sabe usted cuál se su sector? & SI(1)-NS/NR(0) & \parencite{sasb-standards-2023} \\
\hline
2 & ¿Sabe usted cuál es su industria? & SI(1)-NS/NR(0) & \parencite{sasb-standards-2023} \\
\hline
3 & Seleccione los lugares (departamento, municipio, cuenca) que tendrá en cuenta para este diagnóstico & Departamento-Municipio-Cuenca-NS/NR & \parencite{capitals-coalition-2021} \\
\hline
4 & Si su empresa está haciendo esta evaluación por motivos NORMATIVOS seleccione el nivel de cumplimiento normativo & Bajo(1)-Medio(2)-Alto(3)-No se está haciendo por tivos Normativos(0)-NS/NR(0) &  \\
\hline
5 & Si su empresa está haciendo esta evaluación por motivos FINANCIERO seleccioné si  su empresa está haciendo esto para obtener acceso a oportunidades de inversión y/o capital & SI(1)-NO(0)-No se está haciendo por tivos Financieros(0)-NS/NR(0) & \parencite{capitals-coalition-2021} \\
\hline
6 & Seleccione los stakeholders por los cuales está haciendo esta evaluación & Clientes-Proveedores-Trabajadores-Ninguna de las anteriores & \parencite{capitals-coalition-2021} \\
\hline
7 & ¿Su empresa está haciendo esta evaluación por temas de oportunidades o necesidades? & Oportunidades(1)-Necesidades(1)-Ambas(2)-Ninguna de las anteriores(0) &  \\
\hline
8 & ¿Su empresa está haciendo esta evaluación por temas de riesgos físicos? & SI(1)-NO(0)-NS/NR(0) &  \\
\hline
9 & Seleccione los aspectos del negocio que su empresa ha contemplado al pensar en las dependencias e impactos que existen sobre el agua como servicio ecosistémico & Modelo de negocio-Cadena de suministro-Estrategia-Planificación financiera-Planes de transición-Ninguna de las anteriores & \parencite{its-now-for-nature-2023} \\
\hline
10 & ¿Su empresa considera material el uso del agua para sus operaciones? & SI(1)-NO(0)-NS/NR(0) &  \\
\hline
10.1 & ¿Qué grado de materialidad considera que tiene el agua (como servicio ecosistémico) en las operaciones de su empresa? & Muy alto-Alto-Medio-Bajo-Muy Bajo-NS/NR &  \\
\hline
10.2 & Seleccione la importancia de la calidad de agua en el éxito de su negocio & No importa en lo absoluto-No muy importante-Neutral-Importante-Vital-NS/NR & \parencite{disclosure-insight-action-2023} \\
\hline
10.3 & Seleccione la importancia de la cantidad de agua en el éxito de su negocio & No importa en lo absoluto-No muy importante-Neutral-Importante-Vital-NS/NR & \parencite{disclosure-insight-action-2023} \\
\hline
11 & ¿Su empresa tiene políticas internas de gestión integral del agua (GIA)? & SI(1)-NO(0)-NS/NR(0) &  \\
\hline
12 & ¿Su empresa tiene políticas ecosistémicas definidas relacionadas al agua? & SI(1)-NO(0)-NS/NR(0) & \parencite{iso-2004} \\
\hline
12.1 & Seleccione las características que cumplen sus políticas ecosistémicas & Propósito de la empresa (misión, visión, valores)-Contexto de la empresa-Proporciona un marco de referencia-Compromisos para la protección del agua-Compromiso de mejora continua-Ninguna de las anteriores & \parencite{iso-2015} \\
\hline
12.2 & Estaría de acuerdo que sus políticas ecosistémicas siguen esta definición: "La política debería ser apropiada a los impactos sobre el agua de las actividades, productos y servicios de la organización y debería guiar el establecimiento de objetivos y metas" & SI(1)-NO(0)-NS/NR(0) & \parencite{iso-2004} \\
\hline
13 & Indique los objetivos relacionados con la contaminación del agua, las extracciones de agua, WASH u otras categorías relacionadas con el agua que tenga. & La contaminación del agua-Las extracciones de agua-WASH-Otras categorías relacionadas con el agua-Ninguna de las anteriores & \parencite{disclosure-insight-action-2023} \\
\hline
13.1 & ¿Por qué no tienes objetivos(s) relacionados con el agua y cuáles son tus planes para desarrollarlos en el futuro? & Estamos planeando introducir un objetivo en los próximos dos años-Importante, pero no una prioridad comercial inmediata-Falta de recursos internos-Datos insuficientes sobre las operaciones-Ninguna instrucción de la gerencia-Considerado sin importancia & \parencite{disclosure-insight-action-2023} \\
\hline
14 & ¿El concepto de agua está dentro de las estrategias de la empresa? & SI(1)-NO(0)-NS/NR(0) &  \\
\hline
15 & ¿El concepto de agua está dentro del modelo de negocio de la empresa? & SI(1)-NO(0)-NS/NR(0) &  \\
\hline
16 & ¿Usted considera que los cambios e impactos de su empresa sobre los ecosistemas a través del tiempo han seguido las metas y objetivos definidos desde un principio? & SI(1)-NO(0)-NS/NR(0) & \parencite{capitals-coalition-2021} \\
\hline
17 & ¿Se integran las cuestiones relacionadas con el agua en algún aspecto de su plan estratégico de negocios a largo plazo? & Sí, las cuestiones relacionadas con el agua están integradas(2)-No, las cuestiones relacionadas con el agua se examinaron pero no se consideraron estratégicamente pertinentes/significativas(1)-No, las cuestiones relacionadas con el agua aún no se han revisado, pero hay planes para hacerlo en los próximos dos años(1)-No, las cuestiones relacionadas con el agua no fueron revisadas y no hay planes para hacerlo(0) & \parencite{disclosure-insight-action-2023} \\
\hline
18 & ¿Sabe su empresa cuales etapas de la cadena de suministro tiene dependencias y/o impactos? & SI(1)-NO(0)-NS/NR(0) & \parencite{disclosure-insight-action-2023} \\
\hline
18.1 & ¿En qué etapas de la cadena de suministro tiene dependencias e impactos? & Suministro-Fabricación-Distribución-NS/NR & \parencite{its-now-for-nature-2023} \\
\hline
18.2 & ¿Sus proveedores tienen que cumplir con los requisitos relacionados con el agua como parte del proceso de compra de su organización? & Sí, los requisitos relacionados con el agua están incluidos en nuestros contratos de proveedores(3)-Sí, los proveedores tienen que cumplir con los requisitos relacionados con el agua, pero no están incluidos en nuestros contratos de proveedores(2)-No, pero tenemos previsto introducir requisitos relacionados con el agua en los próximos dos años(1)-No, y no tenemos previsto introducir requisitos relacionados con el agua en los próximos dos años(0) & \parencite{disclosure-insight-action-2023} \\
\hline
18.3 & ¿Su empresa conoce y clasifica los lugares y las instalaciones involucradas en su cadena de suministro para identificar donde hay un mayor impacto o dependencia? & Sí, conocemos y calificamos todos los lugares e instalaciones(3)-Sí, conocemos o calificamos todos los lugares e instalaciones(2)-Sí, cocemos o calificamos algunos lugares e instalaciones(1)-No, consideramos que no es importante(0)-No, no tenemos información(0)-No(0) & \parencite{ceres-2023A} \\
\hline
19 & ¿Su organización ya ha realizado algún tipo de evaluación de los riesgos relacionados con el agua? & SI(1)-NO(0)-NS/NR(0) & \parencite{disclosure-insight-action-2023} \\
\hline
19.1 & ¿Por qué su organización no realiza una evaluación de los riesgos relacionados con el agua? & Estamos planeando introducir un proceso de evaluación de riesgos dentro de los próximos dos años-Importante pero no una prioridad comercial inmediata-Considerado sin importancia, explicación proporcionada-Falta de recursos internos-Datos insuficientes sobre las operaciones-Ninguna instrucción de la gerencia-Otros & \parencite{disclosure-insight-action-2023} \\
\hline
20 & ¿Su empresa ha identificado alguna oportunidad relacionada con el agua con el potencial de tener un impacto financiero o estratégico sustancial en su negocio? & Sí, hemos identificado oportunidades, y algunos/ todos se están realizando(2)-Sí, hemos identificado oportunidades, pero no son capaces de realizarlas(1)-No(0) & \parencite{disclosure-insight-action-2023} \\
\hline
20.1 & ¿Por qué en su organización no se considera tener oportunidades relacionadas con el agua? & Existen oportunidades, pero no somos capaces de realizarlas-Existen oportunidades, pero ninguna con potencial para tener un impacto financiero o estratégico sustancial en las empresas-Evaluación en curso-Juzgado como sin importancia-Ninguna instrucción de la gerencia para buscar oportunidades-Aún no se ha evaluado-Otros & \parencite{disclosure-insight-action-2023} \\
\hline
21 & ¿Existe una supervisión a nivel de junta directiva de los asuntos relacionados con el agua dentro de su organización? & SI(1)-NO(0)-NS/NR(0) & \parencite{disclosure-insight-action-2023} \\
\hline
21.1 & Seleccione los stakeholders que usted considera que se deben tener en cuenta durante este diagnóstico & Trabajadores-Alta Gerencia-Terceros-Proveedores-Clientes-Ninguna de las anteriores & \parencite{capitals-coalition-2021} \\
\hline
21.2 & ¿Los reportes relacionados a temas de agua o ecosistemas son expuestos a las altas directivas de la empresa? & SI(1)-NO(0)-NS/NR(0) & \parencite{ceres-2023A} \\
\hline
21.3 & ¿Con qué frecuencias son los informes presentados a las altas directivas? & Mensual(4)-Bimestral(3)-Semestral(2)-Anual(1)-Ninguna de las anteriores(0) & \parencite{ceres-2023A} \\
\hline
21.4 & ¿Existen incentivos para las altas directivas para cumplir con los objetivos relacionados al agua? & SI(1)-NO(0)-NS/NR(0) & \parencite{ceres-2023A} \\
\hline
22 & En el año que se informa, ¿su organización estuvo sujeta a multas, órdenes de cumplimiento y/u otras sanciones por violaciones regulatorias relacionadas con el agua? & SI-NO-NS/NR & \parencite{disclosure-insight-action-2023} \\
\hline
22.1 & Si sí, selección cuales les aplican & Multas-Órdenes de ejecución u otras sanciones-Multas, pero ninguna que se considere significativa-Órdenes de ejecución u otras sanciones, pero ninguna que se considere significativa-Otro-NS/NR & \parencite{disclosure-insight-action-2023} \\
\hline
23 & ¿La empresa participa en actividades que podrían influir directa o indirectamente en las políticas públicas sobre el agua a través de cualquiera de las siguientes actividades? & Sí, compromiso directo con los responsables políticos-Sí, asociaciones comerciales-Sí, financiando organizaciones de investigación-Sí-Ninguna de las anteriores & \parencite{disclosure-insight-action-2023} \\
\hline
24 & ¿La empresa contempla las dependencias e impactos que tiene la extracción y consumo de agua y el vertido de aguas residuales? & SI(1)-NO(0)-NS/NR(0) & \parencite{ceres-2023A} \\
\hline
24.1 & En comparación al año anterior, cuánta agua se ha extraído? & Mucho menos(3)-Menos(2)-Igual(1)-Más(0)-Mucho más(0)-El primer año midiendo(0)-NS/NR(0) & \parencite{disclosure-insight-action-2023} \\
\hline
24.2 & En comparación al año anterior, cuánta agua se ha vertido? & Mucho menos(3)-Menos(2)-Igual(1)-Más(0)-Mucho más(0)-El primer año midiendo(0)-NS/NR(0) & \parencite{disclosure-insight-action-2023} \\
\hline
24.3 & En comparación al año anterior, cuánta agua se ha consuido? & Mucho menos(3)-Menos(2)-Igual(1)-Más(0)-Mucho más(0)-El primer año midiendo(0)-NS/NR(0) & \parencite{disclosure-insight-action-2023} \\
\hline
24.4 & Seleccione la fuente principal de agua que se utiliza para extraer & Agua dulce superficial-Aguas superficiales salobres-Aguas subterráneas-Agua producida/arrastrada-Fuentes de terceros-NS/NR & \parencite{disclosure-insight-action-2023} \\
\hline
24.5 & Seleccione el destino principal del agua verida. & Agua dulce superficial-Aguas superficiales salobres-Aguas subterráneas-Fuentes de terceros-NS/NR & \parencite{disclosure-insight-action-2023} \\
\hline
25 & ¿Alguno de sus productos contiene sustancias clasificadas como peligrosas por un ente regulatorio? & SI-NO-NS/NR & \parencite{disclosure-insight-action-2023} \\
\hline
26 & ¿Su organización identifica y clasifica los contaminantes potenciales del agua asociados con sus actividades que podrían tener un impacto perjudicial en los ecosistemas hídricos o la salud humana? & Sí, identificamos y clasificamos nuestros contaminantes potenciales del agua(1)-No, no identificamos ni clasificamos nuestros contaminantes potenciales del agua(0)-NS/NR(0) & \parencite{disclosure-insight-action-2023} \\
\hline
27 & ¿La empresa asigna recursos financieros y humanos para garantizar el derecho humano al agua y saneamiento, no solo dentro de sus operaciones internas y empleados, sino también incluyendo a proveedores y comunidades afectadas? & SI(1)-NO(0)-NS/NR(0) & \parencite{ceres-2023A} \\
\hline
28 & ¿La empresa proporciona informes detallados sobre la naturaleza exacta de sus compromisos con políticas públicas? & SI(1)-NO(0)-NS/NR(0) & \parencite{ceres-2023A} \\
\hline
29 & ¿La empresa contempla justicia, equidad e inclusividad en sus estrategias para manejar el agua? & SI(1)-NO(0)-NS/NR(0) & \parencite{ceres-2023A} \\
\hline
30 & ¿Su empresa utiliza un precio interno en el agua? & SI(1)-NO(0)-NS/NR(0) & \parencite{ceres-2023A} \\
\hline
31 & La compañía promueve proactivamente el fortalecimiento de la gobernanza del agua, la infraestructura y el acceso equitativo al agua & SI(1)-NO(0)-NS/NR(0) & \parencite{ceres-2023A} \\
\hline
32 & ¿La alta dirección de la empresa crea una cultura y ambiente que estimule roles de liderzago en temas relacionados con el agua? & SI(1)-NO(0)-NS/NR(0) & \parencite{iso-2015} \\
\hline
33 & Seleccione las características que describa la información (relacionada a las dependencias e impactos sobre el agua) que es comunicada interna y externamente  & Transparente-Apropiada-Veraz-Basada en hechos-Completa en su propio contexto-Clara y comprensible-Ninguna de las anteriores & \parencite{iso-2015} \\
\hline
34 & Seleccione los stakeholders que están al tanto de todos los requerimientos normativos relacionados a medidas y políticas relacionadas al agua & Clientes-Proveedores-Trabajadores-Altas directivas-Ninguna de las anteriores & \parencite{iso-2004} \\
\hline
35 & Seleccione el nivel de preparación que tiene la empresa ante nuevos requisitos o modificaciones, de manera que se pueda realizar las acciones apropiadas para seguirlos cumpliendo & Muy alto(4)-Alto(3)-Medio(2)-Bajo(1)-Muy Bajo(0)-No hay ningún tipo de preparación(0) & \parencite{iso-2004} \\
\hline
36 & En el momento de contratación de personal que directamente influya en los procesos con alto impacto en el agua, se hacen preguntas relacionadas al agua & SI(1)-NO(0)-NS/NR(0) & \parencite{iso-2004} \\
\hline
37 & Existe algún tipo de educación o formación para los trabajadores que tiene implicaciones directas con los procesos con alto impacto en el agua & SI(1)-NO(0)-NS/NR(0) & \parencite{iso-2004} \\
\hline
37.1 & Existe algún tipo de informe o seguimiento del avance que ha hecho el personal relacionado al tema del agua & SI(1)-NO(0)-NS/NR(0) & \parencite{iso-2004} \\
\hline
38 & La empresa conoce las causas de las deficiencias y actua sobre ellas para mejorar las condiciones del agua & Sí, conoce las causas y actua sobre ellas(2)-Sí, conoce las causas pero no actua sobre ellas(1)-No, desconoce de las causas y, por lo tanto, no actua sobre ellas(0) & \parencite{iso-2004} \\
\hline
39 & A que demografía afectan sus dependencias/impactos en el ecosistema & Individuos-Comunidades-Organizaciones-NS/NR & \parencite{capitals-coalition-2021} \\
\hline
40 & La empresa tiene en cuenta las opiniones, conocimientos, sugerencias y/o necesidades de las comunidades locales  & SI(1)-NO(0)-NS/NR(0) & \parencite{ceres-2023A} \\
\hline
41 & Con que frecuencia se hace la revisión de los impactos y dependencias sobre el agua en las operaciones del negocio & Mensual(4)-Bimestral(3)-Semestral(2)-Anual(1)-Ninguna de las anteriores(0) & \parencite{iso-1999} \\
\hline
42 & Indique si sus indicadores son más cualitativos o  más cuantitativos & Cualitativos(1)-Cuantitativos(2)-NS/NR(0) & \parencite{capitals-coalition-2021} \\
\hline
43 & Los fuente principal de los datos que la empresa usando proviene & Internamente-Publica-Comercial-NS/NR & \parencite{capitals-coalition-2021} \\
\hline
44 & Seleccione las características de los datos que tiene actualmente la empresa. & Validez científica-Validez estadística-Verificables-Ninguna de las anteriores & \parencite{iso-1999} \\
\hline
45 & Su empresa recolecta indicadores & SI(1)-NO(0)-NS/NR(0) &  \\
\hline
45.1 & Selecciona las categorías de indicadores que su empresa recolecta & Indicadores de la Condicional Ambiental (ICAs)-Indicadores de Desempeño de Gestión (IDGs)-Indicadores de Deempeño Operacional (IDOs)-Ninguna de las anteriores & \parencite{iso-1999} \\
\hline
    
\caption{Preguntas utilizadas para evaluar a la empresa}
\label{tab:preguntas}
\end{longtable}
    
